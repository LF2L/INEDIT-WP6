% Options for packages loaded elsewhere
\PassOptionsToPackage{unicode}{hyperref}
\PassOptionsToPackage{hyphens}{url}
\PassOptionsToPackage{dvipsnames,svgnames,x11names}{xcolor}
%
\documentclass[
  11pt,
]{article}

\usepackage{amsmath,amssymb}
\usepackage{lmodern}
\usepackage{iftex}
\ifPDFTeX
  \usepackage[T1]{fontenc}
  \usepackage[utf8]{inputenc}
  \usepackage{textcomp} % provide euro and other symbols
\else % if luatex or xetex
  \usepackage{unicode-math}
  \defaultfontfeatures{Scale=MatchLowercase}
  \defaultfontfeatures[\rmfamily]{Ligatures=TeX,Scale=1}
\fi
% Use upquote if available, for straight quotes in verbatim environments
\IfFileExists{upquote.sty}{\usepackage{upquote}}{}
\IfFileExists{microtype.sty}{% use microtype if available
  \usepackage[]{microtype}
  \UseMicrotypeSet[protrusion]{basicmath} % disable protrusion for tt fonts
}{}
\makeatletter
\@ifundefined{KOMAClassName}{% if non-KOMA class
  \IfFileExists{parskip.sty}{%
    \usepackage{parskip}
  }{% else
    \setlength{\parindent}{0pt}
    \setlength{\parskip}{6pt plus 2pt minus 1pt}}
}{% if KOMA class
  \KOMAoptions{parskip=half}}
\makeatother
\usepackage{xcolor}
\usepackage[top=30mm,left=20mm]{geometry}
\setlength{\emergencystretch}{3em} % prevent overfull lines
\setcounter{secnumdepth}{5}
% Make \paragraph and \subparagraph free-standing
\ifx\paragraph\undefined\else
  \let\oldparagraph\paragraph
  \renewcommand{\paragraph}[1]{\oldparagraph{#1}\mbox{}}
\fi
\ifx\subparagraph\undefined\else
  \let\oldsubparagraph\subparagraph
  \renewcommand{\subparagraph}[1]{\oldsubparagraph{#1}\mbox{}}
\fi


\providecommand{\tightlist}{%
  \setlength{\itemsep}{0pt}\setlength{\parskip}{0pt}}\usepackage{longtable,booktabs,array}
\usepackage{calc} % for calculating minipage widths
% Correct order of tables after \paragraph or \subparagraph
\usepackage{etoolbox}
\makeatletter
\patchcmd\longtable{\par}{\if@noskipsec\mbox{}\fi\par}{}{}
\makeatother
% Allow footnotes in longtable head/foot
\IfFileExists{footnotehyper.sty}{\usepackage{footnotehyper}}{\usepackage{footnote}}
\makesavenoteenv{longtable}
\usepackage{graphicx}
\makeatletter
\def\maxwidth{\ifdim\Gin@nat@width>\linewidth\linewidth\else\Gin@nat@width\fi}
\def\maxheight{\ifdim\Gin@nat@height>\textheight\textheight\else\Gin@nat@height\fi}
\makeatother
% Scale images if necessary, so that they will not overflow the page
% margins by default, and it is still possible to overwrite the defaults
% using explicit options in \includegraphics[width, height, ...]{}
\setkeys{Gin}{width=\maxwidth,height=\maxheight,keepaspectratio}
% Set default figure placement to htbp
\makeatletter
\def\fps@figure{htbp}
\makeatother
\newlength{\cslhangindent}
\setlength{\cslhangindent}{1.5em}
\newlength{\csllabelwidth}
\setlength{\csllabelwidth}{3em}
\newlength{\cslentryspacingunit} % times entry-spacing
\setlength{\cslentryspacingunit}{\parskip}
\newenvironment{CSLReferences}[2] % #1 hanging-ident, #2 entry spacing
 {% don't indent paragraphs
  \setlength{\parindent}{0pt}
  % turn on hanging indent if param 1 is 1
  \ifodd #1
  \let\oldpar\par
  \def\par{\hangindent=\cslhangindent\oldpar}
  \fi
  % set entry spacing
  \setlength{\parskip}{#2\cslentryspacingunit}
 }%
 {}
\usepackage{calc}
\newcommand{\CSLBlock}[1]{#1\hfill\break}
\newcommand{\CSLLeftMargin}[1]{\parbox[t]{\csllabelwidth}{#1}}
\newcommand{\CSLRightInline}[1]{\parbox[t]{\linewidth - \csllabelwidth}{#1}\break}
\newcommand{\CSLIndent}[1]{\hspace{\cslhangindent}#1}

\makeatletter
\makeatother
\makeatletter
\makeatother
\makeatletter
\@ifpackageloaded{caption}{}{\usepackage{caption}}
\AtBeginDocument{%
\ifdefined\contentsname
  \renewcommand*\contentsname{Table of contents}
\else
  \newcommand\contentsname{Table of contents}
\fi
\ifdefined\listfigurename
  \renewcommand*\listfigurename{List of Figures}
\else
  \newcommand\listfigurename{List of Figures}
\fi
\ifdefined\listtablename
  \renewcommand*\listtablename{List of Tables}
\else
  \newcommand\listtablename{List of Tables}
\fi
\ifdefined\figurename
  \renewcommand*\figurename{Figure}
\else
  \newcommand\figurename{Figure}
\fi
\ifdefined\tablename
  \renewcommand*\tablename{Table}
\else
  \newcommand\tablename{Table}
\fi
}
\@ifpackageloaded{float}{}{\usepackage{float}}
\floatstyle{ruled}
\@ifundefined{c@chapter}{\newfloat{codelisting}{h}{lop}}{\newfloat{codelisting}{h}{lop}[chapter]}
\floatname{codelisting}{Listing}
\newcommand*\listoflistings{\listof{codelisting}{List of Listings}}
\makeatother
\makeatletter
\@ifpackageloaded{caption}{}{\usepackage{caption}}
\@ifpackageloaded{subcaption}{}{\usepackage{subcaption}}
\makeatother
\makeatletter
\@ifpackageloaded{tcolorbox}{}{\usepackage[many]{tcolorbox}}
\makeatother
\makeatletter
\@ifundefined{shadecolor}{\definecolor{shadecolor}{rgb}{.97, .97, .97}}
\makeatother
\makeatletter
\makeatother
\ifLuaTeX
  \usepackage{selnolig}  % disable illegal ligatures
\fi
\IfFileExists{bookmark.sty}{\usepackage{bookmark}}{\usepackage{hyperref}}
\IfFileExists{xurl.sty}{\usepackage{xurl}}{} % add URL line breaks if available
\urlstyle{same} % disable monospaced font for URLs
\hypersetup{
  pdftitle={Livrable WP6},
  colorlinks=true,
  linkcolor={blue},
  filecolor={Maroon},
  citecolor={Blue},
  urlcolor={blue},
  pdfcreator={LaTeX via pandoc}}

\title{Livrable WP6}
\author{}
\date{}

\begin{document}
\maketitle
\ifdefined\Shaded\renewenvironment{Shaded}{\begin{tcolorbox}[borderline west={3pt}{0pt}{shadecolor}, enhanced, sharp corners, breakable, interior hidden, frame hidden, boxrule=0pt]}{\end{tcolorbox}}\fi

\renewcommand*\contentsname{Table of contents}
{
\hypersetup{linkcolor=}
\setcounter{tocdepth}{3}
\tableofcontents
}
\hypertarget{introduction}{%
\section{Introduction}\label{introduction}}

\hypertarget{plastic-issues-for-the-european-union}{%
\section{Plastic Issues for the European
Union}\label{plastic-issues-for-the-european-union}}

Since 1950', our society have gained enormous advantages in terms of
quality of life thanks to the technical development of the development
of plastic and polymer materials. Plastic is a material that is widely
used in our daily lives and plays a fundamental role in industry and
economic development. The plastic material are found in almost all our
products: food packaging, cars, technological tools, clothing, among
others. The main reason is that plastic materials offer a variety of
chemical and mechanical properties to be useful for a wide array of
applications. Plastics are extremely useful, but their mismanagement has
affected the environment and our health. The over-consumption and
especially bad practices (single use, difficulty of reuse, etc.), make
plastics one of the major societal challenges of an ecological
transition that has become imperative. The main problem is the
end-of-life treatment which traditionally uses a centralized system
where plastic waste often has to travel thousands of kilometers\ldots{}
to be incinerated or landfilled. In addition to the energy and
environmental impact of their production, there is also the impact of
the end of life.

Unfortunately, the plastic waste pollution poses a major threat because
of the issue of non-degradability affecting the ecological environments
(\protect\hyperlink{ref-Hopewell2009}{Hopewell et al., 2009};
\protect\hyperlink{ref-Ryberg2019}{Ryberg et al., 2019};
\protect\hyperlink{ref-Thompson2009b}{Thompson et al., 2009}). Indeed,
recycling rates remain small (approx. 14\%) in the plastic packaging
field on a global scale
(\protect\hyperlink{ref-Hahladakis2018}{Hahladakis and Iacovidou,
2018}). Even in Europe, which tends to lead on environmental
stewardship, the recycling rate is about 32.5 wt\%
(\protect\hyperlink{ref-Plastics2019}{Plastics, 2019}). However, these
values consider the amount of plastic waste collected, rather than the
total amount in circulation
(\protect\hyperlink{ref-Kranzinger2018}{Kranzinger et al., 2018}).
Rethinking the development and use of plastics is central to the
circular economy paradigm, to provide less harmful options for the
environment. Thus, more types of plastic packaging are available, but
each reflects diverse circular economy strategies

To tackle this accumulation waste problem, the European strategy for
plastics in the circular economy (CE) is gaining attention in the policy
and business debate surrounding sustainable development of industrial
production (\protect\hyperlink{ref-EC2018}{European Commission, 2018};
\protect\hyperlink{ref-Geissdoerfer2017}{Geissdoerfer et al., 2017}). CE
tackles a central societal issue concerning the current principle
``take, make, dispose'' (linear economy) and its negative effects caused
by the depletion of natural resources, waste generation, biodiversity
loss, pollution (water, air, soil) and non-sustainable economics
(\protect\hyperlink{ref-VanBuren2016}{van Buren et al., 2016}). The
validation (technical, economic, legislative) of waste plastic as a
secondary raw material in industrial processes is considered now a core
target to integrate CE into the plastic value chain
(\protect\hyperlink{ref-Simon2019}{Simon, 2019}). Strategies of open and
closed-loop recycling as well as upcycling and downcycling functionality
approaches can offer paths to validate the secondary raw materials
(\protect\hyperlink{ref-Zhuo2014}{Zhuo and Levendis, 2014}). The
promotion of cross-sectorial valorization of plastic wastes through
Industrial symbiosis approaches seems to be a relevant strategy for the
circular economy strategies of the EU
(\protect\hyperlink{ref-Karaylan2021}{Karayılan et al., 2021})

\hypertarget{context-of-the-3d-printing-of-recycled-plastic-demostrator}{%
\section{Context of the 3D Printing of Recycled Plastic
Demostrator}\label{context-of-the-3d-printing-of-recycled-plastic-demostrator}}

\hypertarget{presentation-of-the-scale-of-the-demostrator-rives-de-meurthe-district-nancy-france}{%
\subsection{Presentation of the scale of the demostrator: Rives de
Meurthe district (Nancy,
France)}\label{presentation-of-the-scale-of-the-demostrator-rives-de-meurthe-district-nancy-france}}

\hypertarget{third-place-octroi-nancy}{%
\subsection{Third place Octroi Nancy}\label{third-place-octroi-nancy}}

\hypertarget{lorraine-fab-living-lab}{%
\subsection{Lorraine Fab Living Lab}\label{lorraine-fab-living-lab}}

\hypertarget{d-printing-of-recycled-plastic-demonstrator-the-green-fablab}{%
\section{3D Printing of recycled plastic demonstrator: the ``Green
FabLab''}\label{d-printing-of-recycled-plastic-demonstrator-the-green-fablab}}

\hypertarget{rationale-for-the-technological-system-of-the-3d-printing-recycling-demonstrator}{%
\subsection{Rationale for the technological system of the 3D printing
recycling
demonstrator}\label{rationale-for-the-technological-system-of-the-3d-printing-recycling-demonstrator}}

Based on the context characteristics of the Lorraine Fab Living Lab® and
the Octroi local ecosystems, the 3D printed demonstrator also called
locally as the ``Green Fablab'', it is an initial demonstrator of the
distributed recycling approach that combines a living lab approach
inside a citizen third place ecosystem.\\
The logistical validation of a closed-loop supply chain for plastic in
DIT approach needs to relied to a certain technology.

The Green FabLab demonstrator aims to experiment the technical
feasibility evaluation of a distributed and local plastic recycling for
3D printing technology in the context of these open innovation spaces
such as fab labs, Hackerspaces or even industrial prototyping zones. In
these geographically distributed spaces, the polymer recycling process
of the surrounding areas (streets, neighbourhood, industrial zones) will
be carried out at small lot sizes minimizing, energy consumptions, and
carbon emissions compared to the tradition centralized systems, as some
researches have already explored this path.

The main purpose of this demonstrator in the INEDIT project is to
understand the conditions under which to deploy a notion of circular
economy with the feedstock of the OS 3D printers. The outputs are, not
only by minimizing use of the environment as a sink for residuals but --
perhaps more importantly -- by minimizing the use of virgin materials.
Hence, the environmental impact of this technology is significantly
reduced. Moreover, taking into account the exponential growth of these
spaces (Fablab, Hakerspace, Makespace), they could help to increase the
efficiency to the problem of polymer recycling through the development
of a distributed recycling approach.

\textbf{The Green FabLab project aims to demonstrate that plastic waste
material can have several uses, and therefore several values, during its
life cycle}. The same material could be recycled and transformed into
new raw material for different products. It is in this spirit that many
associations, SMEs, local authorities and individuals are developing new
local recycling practices that could allow us to aim for an economy that
is more respectful of the environment, fairer for society and more
engaging for local politicians

The Additive manufacturing (AM) -also known as 3D printing- is an
important industrial vector in the and its direct (and distributed)
manufacturing capabilities is becoming a key industrial process that
could play a relevant role in the transition from a linear to circular
economy (\protect\hyperlink{ref-Despeisse2016}{Despeisse et al., 2017}).
AM technologies is expected to transform the production process
(\protect\hyperlink{ref-Chen2017}{Chen et al., 2017};
\protect\hyperlink{ref-Jiang2017}{Jiang et al., 2017};
\protect\hyperlink{ref-Rahman2018}{Rahman et al., 2018}) thanks to its
ability to transform a numerical model into a deposition of material
(points, lines or areas) to create a 3D part
(\protect\hyperlink{ref-Bourell2017}{Bourell et al., 2017}). The
expiration of the first patents has contributed to an increased
interest, creating consumer value and potential for disruption
(\protect\hyperlink{ref-Beltagui2020}{Beltagui et al., 2021};
\protect\hyperlink{ref-West2016a}{West and Kuk, 2016}). In economic
terms, the global additive manufacturing market is expected to reach USD
23.33 billion by 2026 (\protect\hyperlink{ref-ReportsAndData2019}{Data,
2019}). However, determining when and how to take advantage of the
benefits is a challenge for traditional means of production. From a
societal viewpoint, Jiang et al.
(\protect\hyperlink{ref-Jiang2017}{2017}) reported that the product
development could change from traditional stage-gate models to
iterative, agile processes changing the scenario by 2030.

The technical development of INEDIT's demonstrator is based on the
distributed recycling via additive manufacturing (DRAM) approach
(\protect\hyperlink{ref-CruzSanchez2020}{Cruz Sanchez et al., 2020}).
This approach is a major scientific output from the INEDIT project as a
proposition of the future industrial landscape.

DRAM is defined as the use of recycled materials by means of mechanical
recycling process in the 3D printing process chain. In the literature,
DRAM approach emphasizes the technical steps required to reuse plastic
waste through the recycling chains for material-extrusion-based 3D
printing (\protect\hyperlink{ref-CruzSanchez2020}{Cruz Sanchez et al.,
2020}; \protect\hyperlink{ref-Little2020}{Little et al., 2020}). The use
of recycled material, either in the form of raw material or blended with
virgin material, is a method of special interest to contribute to
sustainable manufacturing (\protect\hyperlink{ref-Zhao2018}{Zhao et al.,
2018}).

Figure~\ref{fig-dram} illustrates the conceptual model of DRAM.

\begin{figure}

{\centering \includegraphics[width=0.8\textwidth,height=\textheight]{figures/DRAM-10.png}

}

\caption{\label{fig-dram}DRAM}

\end{figure}

In a general overview, the \textbf{Recovery (I)} phase concerns the
logistic operations to consider to collect the plastic wastes to be
reused in DRAM. The \textbf{Preparation (II)} phase corresponds to the
actions and strategies to identify, separate, sort, size reduce and
clean waste plastic to guarantee adequate quality for DRAM. The
\textbf{Compounding (III)} phase refers to the development of mono- and
composite-materials. The \textbf{Feedstock (IV)} phase identifies the
actions to fabricate the material usable for the printing process,
either filament for Fused Filament Fabrication (FFF) or the particle
size for Fused Granular Fabrication (FGF). The \textbf{Printing (V)}
stage identifies applications and process improvements for the recycled
printed part. The \textbf{Quality (VI)} phase identifies the multi-level
technical characterization performed to the recycled material.

In the DRAM methodology, consumers have an economic incentive to
recycle. This is because they can use their waste as feedstock for a
wide range of consumer products that can be produced for a fraction of
the conventional cost of the equivalent products. Moreover, 3D printing
is especially well suited because it enables the production of parts
with (almost) no waste, and could reduce the waste related to the
material by more than 40 \%, reusing 95\% of the unused material
(\protect\hyperlink{ref-Petrovic2011}{Petrovic et al., 2011}).
Currently, most of the cost of 3D printing is associated with filament
(\protect\hyperlink{ref-Wittbrodt2013}{Wittbrodt et al., 2013}). By
recycling raw materials such as Polylactic acid (PLA), one of the most
frequently used materials in 3D printing, it is possible to reduce the
carbon dioxide emissions that are incurred by transport to landfills or
shipping to customers, offering environmental benefits
(\protect\hyperlink{ref-Santander2020}{Santander et al., 2020}).

A large number of products can already be manufactured with AM, which
affects the geographical spread and density of global value chains
(\protect\hyperlink{ref-Laplume2016}{Laplume et al., 2016}). It is
expected that the reach of AM printable products will be much greater in
the future, as the production of multi-material and built-in
functionalities (e.g.~electronics) will be possible to a large extent.
In addition, the production of spare parts can be carried out on-site,
modifying the role of suppliers in the production lines
(\protect\hyperlink{ref-Zanoni2019}{Zanoni et al., 2019}). Matt et al.
(\protect\hyperlink{ref-Matt2015}{2015}) explored the stages of
distributed model factories and decentralized production types ranging
from distributed capabilities to cloud production. Thus, the need of
transport will be much more carefully because the fact that AM will
enable decentralization of production to localities near customers or in
the most extreme distributed scenario at the customer's premises
(\protect\hyperlink{ref-BonninRoca2019}{Bonnín Roca et al., 2019};
\protect\hyperlink{ref-Petersen2017a}{Petersen and Pearce, 2017};
\protect\hyperlink{ref-Wittbrodt2013}{Wittbrodt et al., 2013}).
Moreover, AM technology makes it possible to reduce market entry
barriers, reduce capital requirements and achieve an efficient minimum
scale of production to promote distributed, flexible forms of production
(\protect\hyperlink{ref-Despeisse2016}{Despeisse et al., 2017}).

This enables an alternative option from an economy-of-scale to an
economy-of-scope, where the products are highly personalized satisfying
niche communities or even individuals
(\protect\hyperlink{ref-Hienerth2014}{Hienerth et al., 2014};
\protect\hyperlink{ref-Petrick2014}{\textbf{Petrick2014?}}). For these
reasons, the AM technology could be a driver for a shift in
manufacturing from globally distributed production to local facilities.
Significant efforts are being made by industry and the scientific
community to move AM techniques from rapid prototyping and tooling
stages towards direct digital manufacturing (DDM)
(\protect\hyperlink{ref-Mueller2012}{Gibson et al., 2010};
\protect\hyperlink{ref-Holmstrom2016}{Holmström et al., 2016}), with the
concomitant environmental and social benefits. Nevertheless, Niaki et
al. (\protect\hyperlink{ref-Niaki2019}{2019}) demonstrated that
environmental and social benefits are not the key preferential factors
in the adoption of AM technologies in different industrial sectors. Only
the economic factor remains relevant in the AM implementation,
considering time- and cost-saving as the most important reasons.

\hypertarget{positionnement-of-use-case-for-omdf-functions}{%
\subsection{Positionnement of Use case for OMDF
Functions}\label{positionnement-of-use-case-for-omdf-functions}}

\hypertarget{hypothesis-of-ul-case-for-deployment-in-reality}{%
\subsection{Hypothesis of UL case for deployment in
reality}\label{hypothesis-of-ul-case-for-deployment-in-reality}}

\hypertarget{technical-characterization-of-the-3d-printing-of-recycled-demonstrator}{%
\subsection{Technical characterization of the 3D printing of recycled
demonstrator}\label{technical-characterization-of-the-3d-printing-of-recycled-demonstrator}}

\hypertarget{recovery-i}{%
\subsubsection{Recovery I}\label{recovery-i}}

The first step in the implementation of the OMDF of the Green Fablab is
the activity of \emph{Recovery I}. This phase aims to establish a
minimal baseline logistic operations to consider to collect the plastic
wastes to be recycled in the process. In the scientific literature, the
recovery is one of the main activities to considered in the recycling
process given that this structures a supply chain that certainly is
variable (i.e.~in fuction of the public supoprt to sort the material in
the specific points of collection). This process is the first step to
create a closed-loop supply network approach for the distributed
manufacturing (\protect\hyperlink{ref-Santander2020}{Santander et al.,
2020}).

The collection tasks consists of collecting plastic waste at different
established points, which are then transported to a treatment center
where it is recycled. The collection and recycling process aims to
generate a recycling micro-network at the local level (neighborhood
scale), which allows the recovery and revaluation of plastic waste
through 3D printing. This allows to save impacts related to the
traditional treatment of plastic waste, as well as to increase the
recycling capacity in the city, giving more independence over the
recycling process. Figure~\ref{fig-recovery} illustrates the process
model considered.

\begin{figure}

{\centering \includegraphics{figures/Recovery.png}

}

\caption{\label{fig-recovery}Recovery processus of the Green Fablab}

\end{figure}

The main difficult relies in the pertinent identification and the
quality state of the plastic waste. Therefore, in the framework of the
INEDIT project, the UL case demonstrator developed a ``smart collector
prototype''. The complete documentation of the technical device can be
found in the following open access reference
(\protect\hyperlink{ref-gabriel2023}{Gabriel and Cruz, 2023}). Given the
possible implementation in other contexts, the source files are shared
in open-source repository with the purpose that open communities to take
advantage the experiences developed at the Université de Lorraine.
Eventually, the open communities can propose improvements and better
versions.

This is a relevant strategy given the cross-line of Industry 4.0 and
circular economy, which is opening up fields such as smart waste
management systems options to improve the effectiveness of different
materials, including plastic waste
(\protect\hyperlink{ref-Ranjbari2021}{Ranjbari et al., 2021}) using
information technology tools with the advent of the Internet of Things
(IoT) Rejeb et al. (\protect\hyperlink{ref-rejeb2022}{2022}). Smart
waste management system (SWMS) consists of public garbage collectors
with embedded technology that is used to monitor real-time level of
garbage bins in public places (\protect\hyperlink{ref-Bano2020}{Bano et
al., 2020}). The interest of this system is to optimize the path for the
garbage collecting van that eventually reduces fuel cost. However, this
work is mainly based on simulation. Therefore, there is an avenue to
simplify experimentation in this domain using common open-source
technology (hardware and software)
(\protect\hyperlink{ref-Pearce2009}{Pearce and Mushtaq, 2009}) to
implement projects that require heavy infrastructure such as routers and
a gateway to deploy in the territory.

The main functional requirement of the smart collector is to collect and
provide data about plastic waste production in order to design a local
and distributed recycling chain of value. However, the smart collector
may be used in various use cases such as:

\begin{itemize}
\tightlist
\item
  Monitoring the quantity of any other product that is collected over a
  large area.\\
\item
  Generating data about behavior to more precisely dimensions public
  infrastructure.\\
\item
  Monitoring the transformation and recycling process inside the
  transformation unit to follow the state and quantity of raw material
  and final product.\\
\item
  Initiating a digitization process in the waste management process as
  the information system element present here is flexible and commonly
  used in various types of projects.
\end{itemize}

\begin{figure}

\begin{minipage}[t]{0.37\linewidth}

{\centering 

\raisebox{-\height}{

\includegraphics{figures/SC/Abstract.png}

}

}

\subcaption{\label{fig-abstract}Description of the developed Smart
collector}
\end{minipage}%
%
\begin{minipage}[t]{0.63\linewidth}

{\centering 

\raisebox{-\height}{

\includegraphics{figures/SC/2022-02-22 smart collectors.png}

}

}

\subcaption{\label{fig-hanno}Prototypes deployed at the local territory}
\end{minipage}%

\caption{\label{fig-elephants}Famous Elephants}

\end{figure}

The device uses a controller compatible with batteries and use WAN
technology to avoid the deployment of routers for data acquisition.
Although using various types of sensors allows us to achieve better
results (\protect\hyperlink{ref-Catania2014}{Catania and Ventura, 2014})
by crossing data, the main indicator remains the weight.

The process illustrated by the Figure~\ref{fig-abstract} can be
described in the as follows:

\begin{enumerate}
\def\labelenumi{\arabic{enumi}.}
\item
  \textbf{Smart Collector installation}: The first step is to identify
  the main actors in the neighborhood through meetings, visits and
  interviews in order to propose integration into the recycling network
  by installing a smart collector on their premises.
\item
  \textbf{Supervision}: The monitoring is done through a dashboard that
  provides direct information sent by the smart collector. This allows
  to know the weight of each installed smart collector, allowing to have
  an approximation of its degree of occupancy.
\item
  \textbf{Receiving and storing plastic waste}: The storage area must be
  organized and functional with respect to the needs of the
  demonstrator.
\item
  \textbf{Plan and execute the collection}:
\end{enumerate}

In the scientific literature, the reverse logistic and closed loop
supply chains have been extensively studied in the scientific
literature. For instance,
(\protect\hyperlink{ref-Santander2021}{\textbf{Santander2021?}})
evaluated the benefits of a near loop and closed loop recycling network
focused on additive manufacturing, mainly producing recycled filament.
The main results show an economic and environmental benefit of sourcing
filament from recycled plastic rather than purchasing exported virgin
filament.

\hypertarget{preparation-ii}{%
\subsubsection{Preparation II}\label{preparation-ii}}

The second phase of the corresponds to the actions and processes to
identify, separate, sort, size reduce and clean waste plastic to
guarantee with the purpose to obtain feedstock material that is adequate
for the distributed recycling process.

The plastic waste preparation process aims to conditioned the collected
plastic to the requirements of 3D printing. For this process 3 main
sub-processes are considered:

\begin{itemize}
\item
  \textbf{Identification and Sorting}: These two processes aim to
  identify the type of plastic given the regular standard for the
  polymer industry. The process of identification and separation of
  plastics is done manually and allows to separate the plastics that can
  be used as raw material for further production processes.
\item
  \textbf{Cleaning}: This process is aims to remove the traces of any
  other substance that may be present in the plastic waste. In this way
  the processing machines will not be exposed to possible anomalies
  linked to material impurities.
\item
  \textbf{Size reduction}: The size reduction process is carried out to
  possible obtain an adequate granulometry. This process allows to adapt
  the plastic waste for the direct injection process and/or the
  extrusion process.
\item
  \textbf{Drying phase}: This step prevents the formation of bubbles in
  the recycled material when it is melted during the following extrusion
  or densification step
  (\protect\hyperlink{ref-Niaounakis2013}{\textbf{Niaounakis2013?}}).
  Moreover, complete elimination of water prevent hydrolytic
  decomposition of the molecular chains during the melting or
  plasticization, so that the treated material has to be as dry as
  possible.
\end{itemize}

\hypertarget{compounding-iii}{%
\subsubsection{Compounding III}\label{compounding-iii}}

The \emph{Compounding} phase is related to the operation, strategies in
the development of composite materials using recycled feedstock intended
to be use in a printing process. There have been several literature
reviews about the technical aspect of composite materials in the
additive manufacturing context Mohan et al.
(\protect\hyperlink{ref-Mohan2017}{2017}).

Special attention has been paid on material extrusion technologies in
the production of polymer/composite feedstocks as it is economical,
environmentally advantageous and adaptable to flexible filament material
(\protect\hyperlink{ref-Singh2017}{Singh et al., 2017}). For example,
Mohan et al. (\protect\hyperlink{ref-Mohan2017}{2017}) presented a
review on composite materials and process parameters optimisation for
the fused deposition modelling (FDM) process for improving the
mechanical properties (i.e.~tensile strength, fatigue). Brenken et al.
(\protect\hyperlink{ref-Brenken2017}{2018}) reported a detailed summary
of mechanical properties of printed parts for different composite
material for fused filament fabrication. Five majors strategies are
elucidated from the literature review such as:

In the context of the Green Fablab demonstrator of INEDIT project, the
focus is to study the 1) mono-recycled material and 2) the
virgin-recycled blend material. The development of recycling niches of
mono-material where the additive manufacturing can be implemented is key
to study. Different studies in laboratory conditions have been made to
show the technical feasibility of recycling. They include HDPE Kreiger
and Pearce (\protect\hyperlink{ref-Kreiger2013}{2013}), Biomass-derived
poly(ethylene-2,5-furandicarboxylate) (PEF)
(\protect\hyperlink{ref-Kucherov2017}{Kucherov et al., 2017}), PLA
(\protect\hyperlink{ref-CruzSanchez2017}{Cruz Sanchez et al., 2017}),
Linear Low Density Polyethylene (LLDPE)/ low density polyethylene
(LDPE), (\protect\hyperlink{ref-Hart2018}{Hart et al., 2018}),
thermoplastic elastomer (TPE) (\protect\hyperlink{ref-Woern2017}{Woern
and Pearce, 2017}). In fact, Hart et al.
(\protect\hyperlink{ref-Hart2018}{2018}) demonstrated the reconstitution
of residual polymeric packaging waste from Meals-Ready-to-Eat (MREs)
generated by soldiers around the world into additively manufactured
appliances. One conclusion of these studies is the positive technical
use of recycled mono-material for additive manufacturing purposes.

However, it has to be highlighted that one major assumption of these
studies relies in that the material used is already sorted, cleaned and
using a same type of discarded product. For INEDIT project, the interest
is to take into account the inner variability that could be in the
recovery process, concerning the type of material given the fact, while
there are seven types of recycling symbols for each type of polymer, one
major constraint in the current systems is that each manufacturing
company have a patented use of the additive in the polymer matrix, in
order to fulfil its initial function of the product.

\hypertarget{feedstock-iv}{%
\subsection{Feedstock IV}\label{feedstock-iv}}

The Feedstock III phase refers to the processes in order to transform
the plastic waste into usable material material for the fabrication
stage. Two outputs are seen in this etape: 1) the filament feedstock and
2) the pellet feedstock. The use of filament or pellet material are in
coherence with the machine process used in the fabrication (cf section
4.4.5). The figure XX present the modeling of the process:

The filament and pellet production process makes it possible to produce
the necessary raw material from plastic waste. The production of these
intermediate products allows the use of different technologies related.
Before using these products (filaments and pellets) it is necessary to
carry out evaluation tests to assess the geometrical characteristics
that are necessary in the printing process.

Figure XX and table present the technical characteristics of the
material equipement

\newpage

\hypertarget{bibliography}{%
\section*{Bibliography}\label{bibliography}}
\addcontentsline{toc}{section}{Bibliography}

\hypertarget{refs}{}
\begin{CSLReferences}{1}{0}
\leavevmode\vadjust pre{\hypertarget{ref-Anderson2017}{}}%
Anderson, I., 2017. Mechanical {Properties} of {Specimens 3D Printed}
with {Virgin} and {Recycled Polylactic Acid}. 3D Printing and Additive
Manufacturing 4, 110--115. \url{https://doi.org/10.1089/3dp.2016.0054}

\leavevmode\vadjust pre{\hypertarget{ref-Bano2020}{}}%
Bano, A., Ud Din, I., Al-Huqail, A.A., 2020. {AIoT-Based Smart Bin} for
{Real-Time Monitoring} and {Management} of {Solid Waste}. Scientific
Programming 2020. \url{https://doi.org/10.1155/2020/6613263}

\leavevmode\vadjust pre{\hypertarget{ref-Beltagui2020}{}}%
Beltagui, A., Sesis, A., Stylos, N., 2021. A bricolage perspective on
democratising innovation: {The} case of {3D} printing in makerspaces.
Technological Forecasting and Social Change 163, 120453.
\url{https://doi.org/10.1016/j.techfore.2020.120453}

\leavevmode\vadjust pre{\hypertarget{ref-BonninRoca2019}{}}%
Bonnín Roca, J., Vaishnav, P., Laureijs, R.E., Mendonça, J., Fuchs,
E.R.H., 2019. Technology cost drivers for a potential transition to
decentralized manufacturing. Additive Manufacturing 28, 136--151.
\url{https://doi.org/10.1016/j.addma.2019.04.010}

\leavevmode\vadjust pre{\hypertarget{ref-Bourell2017}{}}%
Bourell, D., Kruth, J.P., Leu, M., Levy, G., Rosen, D., Beese, A.M.,
Clare, A., 2017. Materials for additive manufacturing. CIRP Annals 66,
659--681. \url{https://doi.org/10.1016/j.cirp.2017.05.009}

\leavevmode\vadjust pre{\hypertarget{ref-Brenken2017}{}}%
Brenken, B., Barocio, E., Favaloro, A., Kunc, V., Pipes, R.B., 2018.
Fused filament fabrication of fiber-reinforced polymers: {A} review.
Additive Manufacturing 21, 1--16.
\url{https://doi.org/10.1016/j.addma.2018.01.002}

\leavevmode\vadjust pre{\hypertarget{ref-Catania2014}{}}%
Catania, V., Ventura, D., 2014. An approch for monitoring and smart
planning of urban solid waste management using smart-{M3} platform, in:
Conference of {Open Innovation Association}, {FRUCT}. {IEEE Computer
Society}, pp. 24--31. \url{https://doi.org/10.1109/FRUCT.2014.6872422}

\leavevmode\vadjust pre{\hypertarget{ref-Chen2017}{}}%
Chen, L., He, Y., Yang, Y., Niu, S., Ren, H., 2017. The research status
and development trend of additive manufacturing technology. The
International Journal of Advanced Manufacturing Technology 89,
3651--3660. \url{https://doi.org/10.1007/s00170-016-9335-4}

\leavevmode\vadjust pre{\hypertarget{ref-CruzSanchez2020}{}}%
Cruz Sanchez, F.A., Boudaoud, H., Camargo, M., Pearce, J.M., 2020.
Plastic recycling in additive manufacturing: {A} systematic literature
review and opportunities for the circular economy. Journal of Cleaner
Production 264, 121602.
\url{https://doi.org/10.1016/j.jclepro.2020.121602}

\leavevmode\vadjust pre{\hypertarget{ref-CruzSanchez2017}{}}%
Cruz Sanchez, F.A., Boudaoud, H., Hoppe, S., Camargo, M., 2017. Polymer
recycling in an open-source additive manufacturing context: {Mechanical}
issues. Additive Manufacturing 17, 87--105.
\url{https://doi.org/10.1016/j.addma.2017.05.013}

\leavevmode\vadjust pre{\hypertarget{ref-ReportsAndData2019}{}}%
Data, R.A., 2019. {ReportsAndData2019}. Additive Manufacturing Market To
Reach USD 23.33 Billion By 2026.

\leavevmode\vadjust pre{\hypertarget{ref-Despeisse2016}{}}%
Despeisse, M., Baumers, M., Brown, P., Charnley, F., Ford, S.J.,
Garmulewicz, A., Knowles, S., Minshall, T.H.W., Mortara, L.,
Reed-Tsochas, F.P., Rowley, J., 2017. Unlocking value for a circular
economy through {3D} printing: {A} research agenda. Technological
Forecasting and Social Change 115, 75--84.
\url{https://doi.org/10.1016/j.techfore.2016.09.021}

\leavevmode\vadjust pre{\hypertarget{ref-EC2018}{}}%
European Commission, 2018. A european strategy for plastics in a
circular economy, COM (2018). {European Commission}, {Brussels}.
\url{https://doi.org/10.1021/acs.est.7b02368}

\leavevmode\vadjust pre{\hypertarget{ref-fatimah2020}{}}%
Fatimah, Y.A., Govindan, K., Murniningsih, R., Setiawan, A., 2020.
Industry 4.0 based sustainable circular economy approach for smart waste
management system to achieve sustainable development goals: {A} case
study of {Indonesia}. Journal of Cleaner Production 269, 122263.
\url{https://doi.org/10.1016/j.jclepro.2020.122263}

\leavevmode\vadjust pre{\hypertarget{ref-gabriel2023}{}}%
Gabriel, A., Cruz, F., 2023. Open source {IoT-based} collection bin
applied to local plastic recycling. HardwareX 13, e00389.
\url{https://doi.org/10.1016/j.ohx.2022.e00389}

\leavevmode\vadjust pre{\hypertarget{ref-Geissdoerfer2017}{}}%
Geissdoerfer, M., Savaget, P., Bocken, N.M.P., Hultink, E.J., 2017. The
{Circular Economy} \textendash{} {A} new sustainability paradigm?
Journal of Cleaner Production 143, 757--768.
\url{https://doi.org/10.1016/j.jclepro.2016.12.048}

\leavevmode\vadjust pre{\hypertarget{ref-Mueller2012}{}}%
Gibson, I., Rosen, D.W., Stucker, B., 2010. Additive {Manufacturing
Technologies}, Assembly Automation. {Springer US}, {Boston, MA}.
\url{https://doi.org/10.1007/978-1-4419-1120-9}

\leavevmode\vadjust pre{\hypertarget{ref-Hahladakis2018}{}}%
Hahladakis, J.N., Iacovidou, E., 2018. Closing the loop on plastic
packaging materials: {What} is quality and how does it affect their
circularity? Science of The Total Environment 630, 1394--1400.
\url{https://doi.org/10.1016/j.scitotenv.2018.02.330}

\leavevmode\vadjust pre{\hypertarget{ref-Hart2018}{}}%
Hart, K.R., Frketic, J.B., Brown, J.R., 2018. Recycling
meal-ready-to-eat ({MRE}) pouches into polymer filament for material
extrusion additive manufacturing. Additive Manufacturing 21, 536--543.
\url{https://doi.org/10.1016/j.addma.2018.04.011}

\leavevmode\vadjust pre{\hypertarget{ref-Hienerth2014}{}}%
Hienerth, C., von Hippel, E., Berg Jensen, M., 2014. User community vs.
Producer innovation development efficiency: {A} first empirical study.
Research Policy 43, 190--201.
\url{https://doi.org/10.1016/j.respol.2013.07.010}

\leavevmode\vadjust pre{\hypertarget{ref-Hofstatter2017}{}}%
Hofstätter, T., Pedersen, D.B., Tosello, G., Hansen, H.N., 2017.
State-of-the-art of fiber-reinforced polymers in additive manufacturing
technologies. Journal of Reinforced Plastics and Composites 36,
1061--1073. \url{https://doi.org/10.1177/0731684417695648}

\leavevmode\vadjust pre{\hypertarget{ref-Holmstrom2016}{}}%
Holmström, J., Holweg, M., Khajavi, S.H., Partanen, J., 2016. The direct
digital manufacturing (r)evolution: Definition of a research agenda.
Operations Management Research 9, 1--10.
\url{https://doi.org/10.1007/s12063-016-0106-z}

\leavevmode\vadjust pre{\hypertarget{ref-Hopewell2009}{}}%
Hopewell, J., Dvorak, R., Kosior, E., 2009. Plastics recycling:
Challenges and opportunities. Philosophical Transactions of the Royal
Society B: Biological Sciences 364, 2115--2126.
\url{https://doi.org/10.1098/rstb.2008.0311}

\leavevmode\vadjust pre{\hypertarget{ref-Jiang2017}{}}%
Jiang, R., Kleer, R., Piller, F.T., 2017. Predicting the future of
additive manufacturing: {A Delphi} study on economic and societal
implications of {3D} printing for 2030. Technological Forecasting and
Social Change 117, 84--97.
\url{https://doi.org/10.1016/j.techfore.2017.01.006}

\leavevmode\vadjust pre{\hypertarget{ref-Karaylan2021}{}}%
Karayılan, S., Yılmaz, Ö., Uysal, Ç., Naneci, S., 2021. Prospective
evaluation of circular economy practices within plastic packaging value
chain through optimization of life cycle impacts and circularity.
Resources, Conservation and Recycling 173, 105691.
\url{https://doi.org/10.1016/j.resconrec.2021.105691}

\leavevmode\vadjust pre{\hypertarget{ref-Kranzinger2018}{}}%
Kranzinger, L., Pomberger, R., Schwabl, D., Flachberger, H., Bauer, M.,
Lehner, M., Hofer, W., 2018. Output-oriented analysis of the wet
mechanical processing of polyolefin-rich waste for feedstock recycling.
Waste Management \& Research 36, 445--453.
\url{https://doi.org/10.1177/0734242X18764294}

\leavevmode\vadjust pre{\hypertarget{ref-Kreiger2013}{}}%
Kreiger, M., Pearce, J.M., 2013. Environmental {Impacts} of {Distributed
Manufacturing} from 3-{D Printing} of {Polymer Components} and
{Products}. MRS Proceedings 1492, 85--90.
\url{https://doi.org/10.1557/opl.2013.319}

\leavevmode\vadjust pre{\hypertarget{ref-Kucherov2017}{}}%
Kucherov, F.A., Gordeev, E.G., Kashin, A.S., Ananikov, V.P., 2017.
Three-{Dimensional Printing} with {Biomass-Derived PEF} for
{Carbon-Neutral Manufacturing}. Angewandte Chemie International Edition
56, 15931--15935. \url{https://doi.org/10.1002/anie.201708528}

\leavevmode\vadjust pre{\hypertarget{ref-Laplume2016}{}}%
Laplume, A.O., Petersen, B., Pearce, J.M., 2016. Global value chains
from a {3D} printing perspective. Journal of International Business
Studies 47, 595--609. \url{https://doi.org/10.1057/jibs.2015.47}

\leavevmode\vadjust pre{\hypertarget{ref-Little2020}{}}%
Little, H.A., Tanikella, N.G., J. Reich, M., Fiedler, M.J., Snabes,
S.L., Pearce, J.M., 2020. Towards {Distributed Recycling} with {Additive
Manufacturing} of {PET Flake Feedstocks}. Materials 13, 4273.
\url{https://doi.org/10.3390/ma13194273}

\leavevmode\vadjust pre{\hypertarget{ref-Matt2015}{}}%
Matt, D.T., Rauch, E., Dallasega, P., 2015. Trends towards {Distributed
Manufacturing Systems} and {Modern Forms} for their {Design}. Procedia
CIRP 33, 185--190. \url{https://doi.org/10.1016/j.procir.2015.06.034}

\leavevmode\vadjust pre{\hypertarget{ref-Mohan2017}{}}%
Mohan, N., Senthil, P., Vinodh, S., Jayanth, N., 2017. A review on
composite materials and process parameters optimisation for the fused
deposition modelling process. Virtual and Physical Prototyping 12,
47--59. \url{https://doi.org/10.1080/17452759.2016.1274490}

\leavevmode\vadjust pre{\hypertarget{ref-Niaki2019}{}}%
Niaki, M.K., Torabi, S.A., Nonino, F., 2019. Why manufacturers adopt
additive manufacturing technologies: {The} role of sustainability.
Journal of Cleaner Production 222, 381--392.
\url{https://doi.org/10.1016/j.jclepro.2019.03.019}

\leavevmode\vadjust pre{\hypertarget{ref-Pearce2009}{}}%
Pearce, J.M., Mushtaq, U., 2009. Overcoming technical constraints for
obtaining sustainable development with open source appropriate
technology. TIC-STH'09: 2009 IEEE Toronto International Conference -
Science and Technology for Humanity 814--820.
\url{https://doi.org/10.1109/TIC-STH.2009.5444388}

\leavevmode\vadjust pre{\hypertarget{ref-Petersen2017a}{}}%
Petersen, E., Pearce, J., 2017. Emergence of {Home Manufacturing} in the
{Developed World}: {Return} on {Investment} for {Open-Source} 3-{D
Printers}. Technologies 5, 7.
\url{https://doi.org/10.3390/technologies5010007}

\leavevmode\vadjust pre{\hypertarget{ref-Petrovic2011}{}}%
Petrovic, V., Vicente Haro Gonzalez, J., Jordá Ferrando, O., Delgado
Gordillo, J., Ramón Blasco Puchades, J., Portolés Griñan, L., 2011.
Additive layered manufacturing: Sectors of industrial application shown
through case studies. International Journal of Production Research 49,
1061--1079. \url{https://doi.org/10.1080/00207540903479786}

\leavevmode\vadjust pre{\hypertarget{ref-Plastics2019}{}}%
Plastics, E., 2019. Plastics - the {Facts} 2019.

\leavevmode\vadjust pre{\hypertarget{ref-Rahman2018}{}}%
Rahman, Z., Barakh Ali, S.F., Ozkan, T., Charoo, N.A., Reddy, I.K.,
Khan, M.A., 2018. Additive {Manufacturing} with {3D Printing}:
{Progress} from {Bench} to {Bedside}. The AAPS Journal 20, 101.
\url{https://doi.org/10.1208/s12248-018-0225-6}

\leavevmode\vadjust pre{\hypertarget{ref-Ranjbari2021}{}}%
Ranjbari, M., Saidani, M., Esfandabadi, Z.S., Peng, W., Lam, S.S.,
Aghbashlo, M., Quatraro, F., Tabatabaei, M., Shams Esfandabadi, Z.,
Peng, W., Lam, S.S., Aghbashlo, M., Quatraro, F., Tabatabaei, M., 2021.
Two decades of research on waste management in the circular economy:
{Insights} from bibliometric, text mining, and content analyses. Journal
of Cleaner Production 314, 128009.
\url{https://doi.org/10.1016/j.jclepro.2021.128009}

\leavevmode\vadjust pre{\hypertarget{ref-rejeb2022}{}}%
Rejeb, A., Suhaiza, Z., Rejeb, K., Seuring, S., Treiblmaier, H., 2022.
The {Internet} of {Things} and the circular economy: {A} systematic
literature review and research agenda. Journal of Cleaner Production
350, 131439. \url{https://doi.org/10.1016/J.JCLEPRO.2022.131439}

\leavevmode\vadjust pre{\hypertarget{ref-Ryberg2019}{}}%
Ryberg, M.W., Hauschild, M.Z., Wang, F., Averous-Monnery, S., Laurent,
A., 2019. Global environmental losses of plastics across their value
chains. Resources, Conservation and Recycling 151, 104459.
\url{https://doi.org/10.1016/j.resconrec.2019.104459}

\leavevmode\vadjust pre{\hypertarget{ref-Santander2020}{}}%
Santander, P., Cruz Sanchez, F.A., Boudaoud, H., Camargo, M., 2020.
Closed loop supply chain network for local and distributed plastic
recycling for {3D} printing: A {MILP-based} optimization approach.
Resources, Conservation and Recycling 154, 104531.
\url{https://doi.org/10.1016/j.resconrec.2019.104531}

\leavevmode\vadjust pre{\hypertarget{ref-Simon2019}{}}%
Simon, B., 2019. What are the most significant aspects of supporting the
circular economy in the plastic industry? Resources, Conservation and
Recycling 141, 299--300.
\url{https://doi.org/10.1016/j.resconrec.2018.10.044}

\leavevmode\vadjust pre{\hypertarget{ref-Singh2017}{}}%
Singh, S., Ramakrishna, S., Singh, R., 2017. Material issues in additive
manufacturing: {A} review. Journal of Manufacturing Processes 25,
185--200. \url{https://doi.org/10.1016/j.jmapro.2016.11.006}

\leavevmode\vadjust pre{\hypertarget{ref-Thompson2009b}{}}%
Thompson, R.C., Moore, C.J., vom Saal, F.S., Swan, S.H., 2009. Plastics,
the environment and human health: Current consensus and future trends.
Philosophical Transactions of the Royal Society B: Biological Sciences
364, 2153--2166. \url{https://doi.org/10.1098/rstb.2009.0053}

\leavevmode\vadjust pre{\hypertarget{ref-VanBuren2016}{}}%
van Buren, N., Demmers, M., van der Heijden, R., Witlox, F., 2016.
Towards a {Circular Economy}: {The Role} of {Dutch Logistics Industries}
and {Governments}. Sustainability 8, 647.
\url{https://doi.org/10.3390/su8070647}

\leavevmode\vadjust pre{\hypertarget{ref-West2016a}{}}%
West, J., Kuk, G., 2016. The complementarity of openness: {How MakerBot}
leveraged {Thingiverse} in {3D} printing 102, 169--181.
\url{https://doi.org/10.1016/j.techfore.2015.07.025}

\leavevmode\vadjust pre{\hypertarget{ref-Wittbrodt2013}{}}%
Wittbrodt, B.T., Glover, A.G., Laureto, J., Anzalone, G.C., Oppliger,
D., Irwin, J.L., Pearce, J.M., 2013. Life-cycle economic analysis of
distributed manufacturing with open-source 3-{D} printers. Mechatronics
23, 713--726. \url{https://doi.org/10.1016/j.mechatronics.2013.06.002}

\leavevmode\vadjust pre{\hypertarget{ref-Woern2017}{}}%
Woern, A., Pearce, J., 2017. Distributed {Manufacturing} of {Flexible
Products}: {Technical Feasibility} and {Economic Viability}.
Technologies 5, 71. \url{https://doi.org/10.3390/technologies5040071}

\leavevmode\vadjust pre{\hypertarget{ref-Zanoni2019}{}}%
Zanoni, S., Ashourpour, M., Bacchetti, A., Zanardini, M., Perona, M.,
2019. Supply chain implications of additive manufacturing: A holistic
synopsis through a collection of case studies. The International Journal
of Advanced Manufacturing Technology 102, 3325--3340.
\url{https://doi.org/10.1007/s00170-019-03430-w}

\leavevmode\vadjust pre{\hypertarget{ref-Zhao2018}{}}%
Zhao, P., Rao, C., Gu, F., Sharmin, N., Fu, J., 2018. Close-looped
recycling of polylactic acid used in {3D} printing: {An} experimental
investigation and life cycle assessment. Journal of Cleaner Production
197, 1046--1055. \url{https://doi.org/10.1016/j.jclepro.2018.06.275}

\leavevmode\vadjust pre{\hypertarget{ref-Zhuo2014}{}}%
Zhuo, C., Levendis, Y.A., 2014. Upcycling waste plastics into carbon
nanomaterials: {A} review. Journal of Applied Polymer Science 131,
n/a--n/a. \url{https://doi.org/10.1002/app.39931}

\end{CSLReferences}



\end{document}
